\documentclass{beamer}

\usetheme{Madrid}
\usecolortheme{default}

\ifPDFTeX
    \usepackage[utf8]{inputenc}
    \usepackage[T5]{fontenc}
    \usepackage[vietnamese]{babel}
\else
    \usepackage{polyglossia}
    \setmainfont{Times New Roman}
    \setsansfont{Arial}
    \setmonofont{Courier New}
    \setdefaultlanguage{vietnamese}
\fi
\usepackage{amsmath}
\usepackage{amssymb}
\usepackage{mathtools}
\usepackage{hyperref}
\usepackage{enumitem}

\title{Recurrent Neural Networks (RNNs)\\and the Exploding \& Vanishing Gradient Problems}
\author{Tran Le Anh Pha \and Nguyen Ngoc Hung \and Le Quang Trung}
\institute{University of Information Technology}
\date{\today}

\begin{document}

\frame{\titlepage}

\begin{frame}{Mục lục}
    \tableofcontents
\end{frame}

\section{Introduction}

\subsection{Sequential Data and Sequence Modeling}

\begin{frame}{Đặc trưng của dữ liệu tuần tự}
    \begin{itemize}
        \item Thứ tự của các phần tử quyết định ý nghĩa: câu chữ, chuỗi thời gian, log hệ thống.
        \item Độ dài có thể biến thiên mạnh và thường được sinh liên tục theo thời gian thực.
        \item Các phụ thuộc xa vẫn ảnh hưởng hiện tại, đòi hỏi mô hình ghi nhớ lịch sử.
    \end{itemize}
    \medskip
    \textbf{Ví dụ:} nhận diện thực thể tên riêng trong câu ``Harry Potter and Hermione Granger invented a new spell.''
\end{frame}

\begin{frame}{Nhu cầu mô hình hóa chuỗi}
    \begin{itemize}
        \item Mô hình phải đọc từng phần tử theo thời gian, cập nhật hidden state và đưa ra dự đoán.
        \item Linh hoạt với các cấu hình input/output: Một-Một, Một-Nhiều, Nhiều-Một, Nhiều-Nhiều.
        \item Chia sẻ tham số giúp số lượng trọng số không phình to khi chuỗi dài.
    \end{itemize}
    \medskip
    Những yêu cầu trên dẫn tới sự ra đời của RNN.
\end{frame}

\subsection{Recurrent Neural Networks Overview}

\begin{frame}{Tổng quan RNN}
    \begin{itemize}
        \item RNN sở hữu hidden state $H_t$ cập nhật tuần tự khi đọc dữ liệu.
        \item Trọng số được tái sử dụng cho mọi thời điểm nên tổng quát tốt trên chuỗi dài.
        \item Ứng dụng: mô hình ngôn ngữ, dịch máy, tổng hợp giọng nói, dự báo chuỗi thời gian.
    \end{itemize}
\end{frame}

\begin{frame}{Ưu và nhược điểm}
    \textbf{Ưu điểm}
    \begin{itemize}
        \item Mô tả phụ thuộc theo thời gian với số tham số cố định.
        \item Có thể xử lý đầu vào/đầu ra độ dài bất kỳ.
    \end{itemize}
    \textbf{Nhược điểm}
    \begin{itemize}
        \item Tính tuần tự cao nên khó song song hóa khi huấn luyện.
        \item Khó học phụ thuộc dài do loss chịu ảnh hưởng mạnh bởi vanishing gradients / exploding gradients.
    \end{itemize}
\end{frame}

\subsection{RNN Architecture and Notation}

\begin{frame}{Ký hiệu cơ bản}
    \begin{itemize}
        \item $\bm{X}_t$: đầu vào tại thời điểm $t$; $\bm{H}_t$: hidden state; $\bm{O}_t$: đầu ra.
        \item Trọng số: $\bm{W}_{xh}$ (input $\rightarrow$ hidden layer), $\bm{W}_{hh}$ (hidden $\rightarrow$ hidden), $\bm{W}_{ho}$ (hidden $\rightarrow$ output layer).
        \item Bias: $\bm{b}_h$, $\bm{b}_o$; hàm kích hoạt $\phi_h$ và $\phi_o$ tuỳ bài toán.
    \end{itemize}
\end{frame}

\begin{frame}{Phương trình kiến trúc}
    \begin{align*}
        \bm{H}_t &= \phi_h\!\left(\bm{X}_t \bm{W}_{xh} + \bm{H}_{t-1} \bm{W}_{hh} + \bm{b}_h\right), \\
        \bm{O}_t &= \phi_o\!\left(\bm{H}_t \bm{W}_{ho} + \bm{b}_o\right).
    \end{align*}
    \begin{itemize}
        \item Hidden state ban đầu $\bm{H}_0$ thường là vector 0 hoặc tham số học được.
        \item Cùng một phép cập nhật áp dụng cho mọi $t$, thể hiện sự chia sẻ tham số theo thời gian.
    \end{itemize}
\end{frame}

\end{document}
