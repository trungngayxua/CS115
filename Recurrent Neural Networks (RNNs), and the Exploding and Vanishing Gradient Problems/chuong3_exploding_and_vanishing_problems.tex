\documentclass{beamer}

\usetheme{Madrid}
\usecolortheme{default}

\usepackage[utf8]{inputenc}
\usepackage[T5]{fontenc}
\usepackage[vietnamese]{babel}
\usepackage{amsmath, amssymb, mathtools}
\usepackage{bm}
\usepackage{graphicx}

\title{Exploding And Vanishing Gradient Problems}
\author{Nhóm trình bày}
\date{\today}

\begin{document}

\frame{\titlepage}

\begin{frame}{Nguồn gốc của vanishing và exploding gradients}

Từ chương trước, gradient theo hidden state tại thời điểm $t$ có dạng:
\begin{equation}
\frac{\partial L}{\partial h_t}
=
\sum_{k=t}^{T}
(W_{hh}^\top)^{k-t}
W_{qh}^\top
\frac{\partial L}{\partial o_k}.
\end{equation}

Do đó, độ lớn của gradient phụ thuộc trực tiếp vào các lũy thừa
của ma trận $W_{hh}^\top$.

\medskip

Vấn đề cốt lõi:
\begin{itemize}
\item Gradient bị nhân lặp lại bởi cùng một ma trận qua nhiều bước thời gian.
\item Hành vi của gradient được quyết định bởi tính chất phổ của $W_{hh}$.
\end{itemize}

\end{frame}

\begin{frame}{Ước lượng độ lớn gradient}

Lấy chuẩn hai vế:
\begin{align}
\left\|\frac{\partial L}{\partial h_t}\right\|
&\le
\sum_{k=t}^{T}
\left\|(W_{hh}^\top)^{k-t}\right\|
\left\|W_{qh}^\top\right\|
\left\|\frac{\partial L}{\partial o_k}\right\|.
\end{align}

Với chuẩn ma trận dưới chuẩn vector:
\begin{equation}
\left\|(W_{hh}^\top)^{k-t}\right\|
\le
\|W_{hh}\|^{k-t}.
\end{equation}

Do đó:
\begin{equation}
\left\|\frac{\partial L}{\partial h_t}\right\|
\lesssim
\sum_{k=t}^{T}
\|W_{hh}\|^{k-t}.
\end{equation}

\end{frame}

\begin{frame}{Điều kiện vanishing và exploding gradients}

Xét giới hạn khi độ dài chuỗi tăng ($T \to \infty$):

\medskip

\begin{itemize}
\item Nếu $\|W_{hh}\| < 1$:
\[
\|W_{hh}\|^{k-t} \to 0
\quad \Rightarrow \quad
\text{gradient vanishing}.
\]

\item Nếu $\|W_{hh}\| > 1$:
\[
\|W_{hh}\|^{k-t} \to \infty
\quad \Rightarrow \quad
\text{gradient exploding}.
\]

\item Nếu $\|W_{hh}\| \approx 1$:
\[
\text{gradient được duy trì ổn định}.
\]
\end{itemize}

\medskip

Trong thực hành, điều kiện này tương đương với
bán kính phổ (spectral radius) của $W_{hh}$.

\end{frame}

\begin{frame}{Tóm tắt vấn đề}

Vanishing và exploding gradients là hệ quả trực tiếp của việc:
\begin{itemize}
\item Nhân lặp cùng một ma trận $W_{hh}$ qua nhiều bước thời gian.
\item Gradient là tổng của các chuỗi lũy thừa ma trận.
\end{itemize}

Do đó, việc huấn luyện RNN chuỗi dài gặp khó khăn
ngay cả trong trường hợp tuyến tính đơn giản.

\end{frame}

\begin{frame}{Giải pháp 1: Gradient Clipping}

Gradient clipping giới hạn độ lớn gradient trong quá trình cập nhật.

\medskip

Cụ thể:
\begin{equation}
g \leftarrow \frac{g}{\max\left(1, \frac{\|g\|}{\tau}\right)},
\end{equation}
trong đó $g$ là gradient và $\tau$ là ngưỡng.

\medskip

Đặc điểm:
\begin{itemize}
\item Không loại bỏ nguyên nhân gốc rễ.
\item Ngăn gradient exploding trong thực tế.
\item Dễ cài đặt, được dùng phổ biến.
\end{itemize}

\end{frame}

\begin{frame}{Giải pháp 2: Long Short-Term Memory (LSTM)}

LSTM thay đổi kiến trúc RNN bằng cách:
\begin{itemize}
\item Tạo đường truyền gradient gần như tuyến tính theo thời gian.
\item Tránh nhân lặp trực tiếp bởi cùng một ma trận.
\end{itemize}

\medskip

Trạng thái cell $c_t$ được cập nhật dưới dạng:
\begin{equation}
c_t = f_t \odot c_{t-1} + i_t \odot \tilde{c}_t,
\end{equation}
trong đó $f_t$ là forget gate.

\medskip

Gradient có thể đi xuyên qua nhiều bước thời gian
mà không bị suy giảm hoặc bùng nổ nghiêm trọng.

\end{frame}

\end{document}
